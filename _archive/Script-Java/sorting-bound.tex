\section{Eine untere Schranke f\"ur Sortier-Algorithmen}
Wir wollen in diesem Abschnitt zeigen, dass jeder Sortier-Algorithmus, der in der Lage
ist,  eine beliebige Folge von Elementen zu sortieren, mindesten die Komplexit\"at
 $\Oh\bigl(n \cdot \ln(n)\bigr)$ haben muss.  Dabei setzen wir voraus, dass die einzelnen Elemente nur mit
 Hilfe des Operators $<$ verglichen werden k\"onnen und wir setzen weiter voraus,
dass wir eine Folge von $n$ verschiedenen Elementen haben, die wir sortieren wollen.  Wir betrachten zun\"achst
eine Folge von zwei Elementen: $[a_1, a_2]$.  Um diese Folge zu sortieren, reicht ein Vergleich aus, denn es
gibt nur zwei M\"oglichkeiten, wie diese beiden Elemente sortiert sein k\"onnen:  
\begin{enumerate}
\item Falls $a_1 < a_2$ ist, dann ist $[a_1, a_2]$ aufsteigend sortiert.
\item Falls $a_2 < a_1$ ist, dann ist $[a_2, a_1]$ aufsteigend sortiert.
\end{enumerate}
Falls wir eine Folge von drei Elementen $[a_1,a_2,a_3]$ haben, so gibt es bereits $6$ M\"oglichkeiten, diese
anzuordnen:
\\[0.2cm]
\hspace*{1.3cm}
$[a_1,a_2,a_3]$, \quad
$[a_1,a_3,a_2]$, \quad
$[a_2,a_1,a_3]$, \quad
$[a_2,a_3,a_1]$, \quad
$[a_3,a_1,a_2]$, \quad
$[a_3,a_2,a_1]$. \quad
\\[0.2cm]
Im allgemeinen gibt es $n! = \prod\limits_{i=1}^n i$ verschiedene M\"oglichkeiten, die Elemente einer
$n$-elementigen Folge $[a_1,a_2,\cdots,a_n]$ anzuordnen. Dies l\"asst sich am einfachsten durch Induktion
nachweisen: 
\begin{enumerate}
\item Offenbar gibt es genau eine M\"oglichkeit, eine Folge von einem Element anzuordnen.
\item Um eine Folge von $n+1$ Elementen anzuordnen haben wir $n+1$ M\"oglichkeiten, das erste Element
      der Folge auszusuchen.  In jedem dieser F\"alle haben wir dann nach Induktions-Voraussetzung
      $n!$ M\"oglichkeiten, die restlichen Elemente anzuordnen, so dass wir insgesamt auf 
      $(n+1) \cdot n! = (n+1)!$ verschiedene Anordnungsm\"oglichkeiten kommen.
\end{enumerate}
Wir \"uberlegen jetzt umgekehrt, aus wievielen M\"oglichkeiten wir mit $k$ verschiedenen Tests ausw\"ahlen k\"onnen.
\begin{enumerate}
\item Offenbar k\"onnen wir mit einem Test aus zwei M\"oglichkeiten ausw\"ahlen.
\end{enumerate}

%%% Local Variables: 
%%% mode: latex
%%% TeX-master: "algorithmen"
%%% End: 
