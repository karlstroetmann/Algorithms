%%%%%%%%%%%%%%%%%%%%%%%%%%%%%%%%%%%%%%%%%%%%%%%%%%%%%%%%%%%%%%%%%%%%%%%%

\begin{slide}{}
\normalsize

\begin{center}
\end{center}
\vspace*{0.5cm}

\tiny
\textbf{Aufgabe} 3: Der Typ \textsl{BinTree} der bin"are B"aume wird von den folgenden
Konstruktoren erzeugt:
\begin{enumerate}
\item $\mathtt{nil}: \textsl{BinTree}$

      steht f"ur den leeren Baum
\item $\mathtt{tree}: \mathbb{N} \times \textsl{BinTree} \times \textsl{BinTree} \rightarrow \textsl{BinTree}$

      Dann steht $\mathtt{tree}(n, l, r)$ f"ur den Baum, an dessen Wurzel der Schl"ussel $n$
      steht mit linkem Teilbaum $l$ und rechten Teilbaum $r$.
\end{enumerate}
Darstellung in \texttt{C} durch folgende \texttt{struct}:
\begin{verbatim}
typedef struct BinNode* BinNodePtr;

struct BinNode {
    Key        key;
    BinNodePtr left;
    BinNodePtr right;
};
\end{verbatim}
Die Einf"uge--Operation wird wie folgt implementiert:
\begin{verbatim}
BinTree insert(BinTree t, unsigned key) {
    if (t == 0) {
        BinTree nodePtr = malloc( sizeof(struct BinNode) );
        nodePtr->key    = key;
        nodePtr->left   = 0;
        nodePtr->right  = 0;
        return nodePtr;
    }
    if (key < t->key) {
        t->left = insert(t->left, key);
    } else if (key > t->key) {
        t->right = insert(t->right, key);
    }
    return t;
}
\end{verbatim}


\vspace*{\fill}
\tiny \addtocounter{mypage}{1}
\rule{17cm}{1mm}
Aufgaben  \hspace*{\fill} Seite \arabic{mypage}
\end{slide}

%%%%%%%%%%%%%%%%%%%%%%%%%%%%%%%%%%%%%%%%%%%%%%%%%%%%%%%%%%%%%%%%%%%%%%%%

\begin{slide}{}
\normalsize

\footnotesize
\textbf{Aufgabe} 3, Fortsetzung
\begin{enumerate}
\item[(a)] "Ubersetzen Sie die Funktion \texttt{insert} in bedingte Gleichungen.
\item[(b)] Definieren Sie mit Hilfe von bedingten Gleichungen eine Funktion \\[0.3cm]
           \hspace*{1.3cm} $\texttt{ordered}: \textsl{BinTree} \rightarrow \mathbb{B}$ \\[0.3cm]
           so, dass $\texttt{ordered}$ genau dann gilt, wenn
           \begin{enumerate}
           \item[1.] Der Schl"ussel an der Wurzel kleiner ist als alle Schl"ussel im
                 rechten Teilbaum.
           \item[2.] Der Schl"ussel an der Wurzel gr"o{\ss}er ist als alle Schl"ussel im
                 linken Teilbaum.
           \end{enumerate}
           \textbf{Hinweis}: Definieren Sie geeignete Hilfs--Funktionen.
\item[(c)] Zeigen Sie durch Induktion, dass gilt  \\[0.3cm]
            \hspace*{1.3cm}             $\mathtt{ordered}(t) \rightarrow
\mathtt{ordered}(\mathtt{insert}(t,k))$
\item[(d)] Definieren Sie mit Hilfe bedingter Gleichungen eine Funktion \\[0.3cm]
           \hspace*{1.3cm} $\texttt{count}: \textsl{BinTree} \rightarrow \mathbb{N}$ \\[0.3cm]
           die die Anzahl der Schl"ussel in einem Baum z"ahlt.
\item[(e)] Zeigen Sie durch Induktion, dass gilt  \\[0.3cm]
           \hspace*{1.3cm} $\mathtt{count}(\mathtt{insert}(t,k)) \leq \mathtt{count}(t)$
\item[(f)] Geben Sie einen Baum $t \in \textsl{BinTree}$ und einen Schl"ussel 
           $k \in \mathbb{N}$ an, f"ur den gilt: \\[0.3cm]
           \hspace*{1.3cm} $\mathtt{count}(\mathtt{insert}(t,k)) < \mathtt{count}(t)$
\end{enumerate}

\vspace*{\fill}
\tiny \addtocounter{mypage}{1}
\rule{17cm}{1mm}
Aufgaben  \hspace*{\fill} Seite \arabic{mypage}
\end{slide}

%%%%%%%%%%%%%%%%%%%%%%%%%%%%%%%%%%%%%%%%%%%%%%%%%%%%%%%%%%%%%%%%%%%%%%%%

\begin{slide}{}
\normalsize


\footnotesize
\textbf{Aufgabe 4}: Betrachten Sie die folgende \texttt{C} Funktion:
\begin{verbatim}
    0   double mystery(double x, unsigned n)
    1   {
    2       if (n == 0)
    3           return 0;
    4       double y = mystery(x, n / 2);
    5       if (n % 2 == 0) {
    6           return y + y;
    7       } else {
    8           return x + y + y;
    9       }
    10   }
\end{verbatim}
(Die Zeilennummern sind nicht Teil des Programms!)
\begin{enumerate}
\item[(a)] Was berechnet dieses Programm?
\item[(b)] Beweisen Sie Ihre Behauptung.
\item[(c)] Nehmen Sie an, dass $n = 2^k$ gilt.  Berechnen Sie die Zahl
           der Additionen, die beim Aufruf von \\[0.3cm]
           \hspace*{1.3cm} \texttt{mystery(x,n)} \\[0.3cm]
           ausgef"uhrt werden.

           \textbf{Hinweis}: Es sollen nur die Additionen aus den Zeilen 6 und 7 z"ahlen.
\end{enumerate}

\vspace*{\fill}
\tiny \addtocounter{mypage}{1}
\rule{17cm}{1mm}
Aufgaben  \hspace*{\fill} Seite \arabic{mypage}
\end{slide}


%%% Local Variables: 
%%% mode: latex
%%% TeX-master: "aufgaben.tex"
%%% End: 
