\documentclass{article}
\usepackage{german}
\usepackage[latin1]{inputenc}
\usepackage{a4wide}
\usepackage{amssymb}
\usepackage{fancyvrb}
\usepackage{alltt}
\usepackage{fleqn}

\usepackage{hyperref}
\usepackage[all]{hypcap}
\hypersetup{
	colorlinks = true, % comment this to make xdvi work
	linkcolor  = blue,
	citecolor  = red,
        filecolor  = blue,
        urlcolor   = [rgb]{0.5, 0.4, 0.0},
	pdfborder  = {0 0 0} 
}

\renewcommand{\labelenumi}{(\alph{enumi})}

\setlength{\mathindent}{1.3cm}

\pagestyle{empty}

\begin{document}
\noindent
{\Large \textbf{Aufgaben-Blatt}: \emph{Arithmetische Ausdr�cke}}

\vspace{0.5cm}


\noindent
\textbf{Aufgabe 1}: 
Erweitern Sie das Programm zur Berechnung arithmetischer Ausdr�cke in mehreren 
Schritten wie folgt:
\begin{enumerate}
\item Ersetzen Sie den Datentyp \texttt{BigInteger} durch \texttt{Double}.
\item Erweitern Sie das Programm so, dass auch die
      Funktionen ``\texttt{sqrt}'', ``\texttt{exp}'', ``\texttt{log}'', ``\texttt{sin}'',
      ``\texttt{cos}'', ``\texttt{tan}'' und die zugeh�rigen Umkehrfunktionen 
      ``\texttt{asin}'', ``\texttt{acos}'' und ``\texttt{atan}''
      verwendet werden
      k�nnen.
\item Erweitern Sie das Programm so, dass auch die Konstanten \texttt{Pi} und
      \texttt{e} in Formeln verwendet werden k�nnen.
\end{enumerate}
Sie finden dieses Programm in dem Ordner
\\[0.2cm]
\hspace*{0.3cm}
\href{https://github.com/karlstroetmann/Algorithms/tree/master/Java/Calculator}{\texttt{https://github.com/karlstroetmann/Algorithms/tree/master/Java/Calculator/}}.
\vspace{0.5cm}

\noindent
\textbf{Aufgabe 2}: 
Erweitern Sie das Programm aus Aufgabe 1 so, dass Sie es zur Berechnung der Nullstelle
einer Funktion einsetzen k�nnen.  Dazu m�ssen die arithmetischen
Ausdr�cke nun auch eine Variable $x$ enthalten d�rfen.  
Berechnen Sie die Nullstelle mit
Hilfe des Bisektions-Verfahrens.  Ein Java Programm zum Bisektions-Verfahren
 finden Sie auf meiner Webseite in dem oben angegebenen Ordner.
Dieses Programm setzt voraus, dass die Klasse \texttt{Calculator} eine Methode der Form
\\[0.2cm]
\hspace*{1.3cm}
\texttt{public double evaluate(double x) \{ $\cdots$ \} }
\\[0.2cm]
enth�lt.  Diese Methode wertet den vom Benutzer vorher eingegebenen arithmetischen
Ausdruck, der die Variable \texttt{x} enth�lt, dadurch aus, dass f�r \texttt{x}
der Wert eingesetzt wird, welcher der Methode \texttt{evaluate()} als Argument �bergeben
worden ist.

Der Aufruf des Programms soll dann in der folgenden Form geschehen:
\\[0.2cm]
\hspace*{1.3cm}
\texttt{java Bisection} \textsl{left} \textsl{right} \textsl{expr}
\\[0.2cm]
Hierbei bezeichnen  \textsl{left} und \textsl{right} Intervall-Grenzen und \textsl{expr}
bezeichnet einen arithmetischen Ausdruck, der die Variable \texttt{x} enth�lt und als
Funktion dieser Variablen interpretiert wird.  Der Aufruf soll innerhalb des
durch die Grenzen spezifizierten Intervalls mit Hilfe des Bisektions-Verfahrens
nach einer Nullstelle der Funktion suchen.  Beispielsweise soll der Aufruf
\\[0.2cm]
\hspace*{1.3cm}
\texttt{0 2 x \symbol{94} 2 - 2}
\\[0.2cm]
die Nullstelle der Funktion $x \mapsto x^2 - 2$ im Intervall $[0, 2]$ suchen.
\vspace{0.3cm}

\noindent
\textbf{Hinweis}:  Es ist zweckm��ig, in der Klasse \texttt{Calculator} eine
Member-Variable \texttt{mValueOfX} zu definieren, die bei jedem Aufruf der Methode
\texttt{evaluate()} auf den Wert gesetzt wird, der der Methode als Argument �bergeben wird.
\vspace{0.3cm}

\noindent
Testen Sie Ihr Programm, indem Sie die Nullstelle der Funktion 
\[ f: \mathbb{R} \rightarrow \mathbb{R} \]
die durch 
\[ f(x) = x^2 - 2 \]
definiert ist, in dem Intervall $[0,2]$ berechnen.
\end{document}

%%% Local Variables: 
%%% mode: latex
%%% TeX-master: t
%%% End: 
