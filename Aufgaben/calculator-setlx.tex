\documentclass{article}
\usepackage{german}
\usepackage[latin1]{inputenc}
\usepackage{a4wide}
\usepackage{amssymb}
\usepackage{fancyvrb}
\usepackage{alltt}
\usepackage{fleqn}

\usepackage{hyperref}
\usepackage[all]{hypcap}
\hypersetup{
	colorlinks = true, % comment this to make xdvi work
	linkcolor  = blue,
	citecolor  = red,
        filecolor  = blue,
        urlcolor   = [rgb]{0.5, 0.4, 0.0},
	pdfborder  = {0 0 0} 
}

\renewcommand{\labelenumi}{(\alph{enumi})}

\setlength{\mathindent}{1.3cm}

\pagestyle{empty}

\begin{document}
\noindent
{\Large \textbf{Aufgaben-Blatt}: \emph{Arithmetische Ausdr�cke}}

\vspace{0.5cm}


\noindent
\textbf{Aufgabe 1}: 
Erweitern Sie das in der Vorlesung diskutierte Programm zur Berechnung arithmetischer Ausdr�cke in mehreren 
Schritten wie folgt:
\begin{enumerate}
\item Erweitern Sie das Programm so, dass auch die
      Funktionen ``\texttt{sqrt}'', ``\texttt{exp}'', ``\texttt{log}'', ``\texttt{sin}'',
      ``\texttt{cos}'', ``\texttt{tan}'' und die zugeh�rigen Umkehrfunktionen 
      ``\texttt{asin}'', ``\texttt{acos}'' und ``\texttt{atan}''
      verwendet werden k�nnen.
\item Erweitern Sie das Programm so, dass auch die Konstanten \texttt{Pi} und
      \texttt{e} in Formeln verwendet werden k�nnen.
\end{enumerate}
Sie finden das in der Vorlesung diskutierte Programm im Netz in dem Ordner
\\[0.2cm]
\hspace*{0.3cm} 
\href{https://github.com/karlstroetmann/Algorithms/tree/master/SetlX/Calculator}{\texttt{https://github.com/karlstroetmann/Algorithms/tree/master/SetlX/Calculator/}}.
\vspace{0.5cm}

\noindent
\textbf{Aufgabe 2}: 
Erweitern Sie das Programm aus Aufgabe 1 so, dass Sie es zur Berechnung der Nullstelle
einer Funktion einsetzen k�nnen.  Dazu m�ssen die arithmetischen
Ausdr�cke nun auch eine Variable $x$ enthalten d�rfen.  
Berechnen Sie die Nullstelle mit Hilfe des Bisektions-Verfahrens, das wir in der Vorlesung �ber
Analysis diskutiert haben.
Sie finden das in der Vorlesung diskutierte Programm im Netz unter
\\[0.2cm]
\hspace*{0.3cm} 
\href{https://github.com/karlstroetmann/Analysis/tree/master/SetlX/bisection.stlx}{\texttt{https://github.com/karlstroetmann/Analysis/tree/master/SetlX/bisection.stlx}}.
\vspace{0.5cm}


\noindent
\textbf{Hinweis}: Syntaktisch k�nnen Sie die Variable $x$ wie eine Konstante behandeln.
\vspace{0.5cm}

\noindent
Testen Sie Ihr Programm, indem Sie die Nullstelle der Funktion 
\[ f: \mathbb{R} \rightarrow \mathbb{R} \]
die durch 
\[ f(x) = x^2 - 2 \]
definiert ist, in dem Intervall $[0,2]$ auf 100 Stellen nach dem Komma berechnen.  Benutzen Sie
hierf�r die Funktion
\\[0.2cm]
\hspace*{1.3cm}
\texttt{nDecimalPlaces(q, n)}.
\\[0.2cm]
Diese Funktion gibt die rationale Zahl $q$ mit $n$ Stellen hinter dem Komma aus.
\end{document}

%%% Local Variables: 
%%% mode: latex
%%% TeX-master: t
%%% End: 
