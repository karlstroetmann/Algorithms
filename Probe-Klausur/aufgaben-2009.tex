\documentclass{article}
\usepackage{german}
\usepackage[latin1]{inputenc}

\usepackage{a4wide}
\usepackage{amssymb}
\usepackage{epsfig}

\setlength{\textwidth}{15cm}

\newcommand{\cq}{\symbol{34}}
\newcommand{\Ll}{{\cal L}}
\newcommand{\Rl}{{\cal R}}
\newcommand{\NS}{{\cal N\!S}}
\newcommand{\cl}[1]{{\cal #1}}
\renewcommand{\labelenumi}{(\alph{enumi})}

\begin{document}

\noindent
{\large Aufgaben zur Vorlesung  ``{\sl Algorithmen und Datenstrukturen}''}
\vspace{0.5cm}


\noindent
\textbf{Aufgabe 1}:
\begin{enumerate}
\item L\"osen Sie die Rekursions-Gleichung \\[0.2cm]
      \hspace*{1.3cm} $a_{n+2} = a_n + 2$ \\[0.2cm]
      f\"ur die Anfangs-Bedingungen $a_0 = 2$ und $a_1 = 1$.
      \hspace*{\fill} (10 Punkte)
\item L\"osen Sie die Rekursions-Gleichung \\[0.2cm]
      \hspace*{1.3cm} $a_{n+2} = 2 \cdot a_n - a_{n+1}$ \\[0.2cm]
      f\"ur die Anfangs-Bedingungen $a_0 = 0$ und $a_1 = 3$.
      \hspace*{\fill} (10 Punkte)
\end{enumerate}
\vspace{0.3cm}

\noindent
\textbf{Aufgabe 2}:
Der geordnete bin\"are Baum $t$ sei durch den folgenden Term definiert,
wobei zur Vereinfachung auf die Angabe der Werte, die mit den Schl\"usseln
assoziiert sind, verzichtet wurde.
\\[0.2cm]
\hspace*{1.3cm}
$t = \textsl{node}(11, \textsl{node}(10, \textsl{nil}, \textsl{nil}), \textsl{node}(15, \textsl{nil}, 
  \textsl{node}(18, \textsl{node}(17, \textsl{nil}, \textsl{nil}), 
  \textsl{node}(24, \textsl{nil}, \textsl{nil}))))
$
%\hspace*{1.3cm}\epsfig{file=aufgabe2,scale=0.5}
\begin{enumerate}
\item F\"ugen Sie in diesem Baum den Schl\"ussel \texttt{16} ein und geben Sie den
      resultierenden Term an.  \\ 
      \hspace*{\fill} (3 Punkte)
\item F\"ugen Sie in dem in Teil (a) berechneten Baum den Schl\"ussel \texttt{13} ein und
      geben Sie den resultierenden Baum an.     
      \hspace*{\fill} (3 Punkte)
\item L\"oschen Sie aus dem in Teil (b) berechneten Baum den Schl\"ussel \texttt{15} und
      geben Sie den resultierenden Baum. 
      \hspace*{\fill} (4 Punkte)
\end{enumerate}
\textbf{Hinweis}: Bei der L\"osung dieser und der folgenden Aufgabe sind selbstverst\"andlich
die in der Vorlesung vorgestellten Algorithmen zu verwenden.
\vspace{0.2cm}

\noindent
\textbf{Aufgabe 3}: Der AVL-Baum $t$ sei durch den folgenden Term gegeben,
wobei zur Vereinfachung auf die Angabe der Werte, die mit den Schl\"usseln
assoziiert sind, verzichtet wurde.
\\[0.2cm]
\hspace*{1.3cm}
$t = \textsl{node}(17, \textsl{node}(8, \textsl{node}(2, \textsl{nil}, \textsl{nil}),
 \textsl{node}(10, \textsl{nil}, \textsl{nil})), \textsl{node}(23,\textsl{nil},\textsl{nil}))$
%\epsfig{file=avl1,scale=0.5}
\begin{enumerate}
\item F\"ugen Sie  in diesem Baum den Schl\"ussel \texttt{13} ein und geben Sie den
      resultierenden Baum an.   \\[0.2cm]
      \hspace*{\fill} (4 Punkte)
\item F\"ugen Sie in dem Baum aus Teil den Schl\"ussel \texttt{15} ein und geben Sie den
      resultierenden Baum an.
      \hspace*{\fill} (3 Punkte)
\item Entfernen Sie den Schl\"ussel 2 aus dem unter Teil (b) berechneten Baum und geben Sie
      den resultierenden Baum an.
      \hspace*{\fill} (4 Punkte)
\end{enumerate}

\noindent
\textbf{Aufgabe 4}:  Betrachten Sie das folgende \texttt{C}-Programm:
\begin{verbatim}
    unsigned sum(unsigned n) {
        if (n == 0)
            return 0;
        return n + sum(n-1);
    }
\end{verbatim}
Weisen Sie mit Wertverlaufs-Induktion nach, dass die Funktion $\textsl{sum}()$  die
folgende Spezifikation erf\"ullt:
\\[0.2cm]
\hspace*{1.3cm}
$\textsl{sum}(n) = \frac{1}{2} \cdot n \cdot (n + 1)$.
\vspace{0.3cm}
\pagebreak

\noindent
\textbf{Aufgabe 5}:  Betrachten Sie das folgende \texttt{C}-Programm:
\begin{verbatim}
    unsigned sum(unsigned n) {
        unsigned i = 0;
        unsigned s = 0;
        while (i <= n) {
            s = i + s;
            i = i + 1;
        }
        return s;
    }
\end{verbatim}
Die Funktion $\textsl{sum}()$  soll die folgende Spezifikation erf\"ullen:
\\[0.2cm]
\hspace*{1.3cm}
$\textsl{sum}(n) = \frac{1}{2} \cdot n \cdot (n + 1)$
\begin{enumerate}
\item Weisen Sie mit Hilfe des Hoare-Kalk\"uls nach, dass das Programm korrekt ist.
\item Beweisen Sie mit Hilfe der Methode der symbolischen Programm-Ausf\"uhrung nach,
      dass das Programm korrekt ist.
\end{enumerate}
\vspace{0.3cm}

\noindent
\textbf{Aufgabe 6}:
\begin{enumerate}
\item Zeigen Sie $\log_2(n) \in \mathcal{O}\bigl(\ln(\sqrt{n})\bigr)$. \hspace*{\fill} (4  Punkte)
\item Zeigen Sie $\bigl|\sin(n)\bigr| \in \mathcal{O}\bigl(1\bigr)$. \hspace*{\fill} (4 Punkte)
\item Es sei 
      $f(n) := \biggl(\sum\limits_{i=1}^n \frac{1}{i}\biggr) - \ln(n)$.
      Zeigen Sie $f(n)\in \mathcal{O}\bigl(1\bigr)$. \hspace*{\fill} (12 Punkte)
      
      \textbf{Hinweis}: Zeigen Sie die Ungleichung
      \\[0.2cm]
      \hspace*{1.3cm}
      $0 \leq \biggl(\sum\limits_{i=1}^n \frac{1}{i}\biggr) - \ln(n) \leq 1$
      \\[0.2cm]
      indem Sie die Summe $\sum\limits_{i=1}^n \frac{1}{i}$ durch geeignete Integrale absch\"atzen.
\end{enumerate}
\vspace{0.3cm}


\noindent
\textbf{Aufgabe 7}:  Im Abschnitt 8.2 des Skriptes
werden Gleichungen angegeben, die das Einf\"ugen und L\"oschen in einem Heap beschreiben.
In diesem Zusammenhang sollen Sie in dieser Aufgabe  einige zus\"atzliche Methoden auf
bin\"aren B\"aumen durch bedingte Gleichungen spezifizieren.
\begin{enumerate}
\item Spezifizieren Sie eine Methode \textsl{isHeap}, so
      dass f\"ur einen bin\"aren Baum $b \in \mathcal{B}$ der Ausdruck 
      $b.\mathtt{isHeap}()$ genau dann den Wert $\mathtt{true}$ hat, wenn $b$ die
      \emph{Heap-Bedingung} erf\"ullt.  \hspace*{\fill} (10 Punkte)
\item Implementieren Sie eine Methode \textsl{isBalanced}, so
      dass f\"ur einen bin\"aren Baum $b \in \mathcal{B}$ der Ausdruck 
      $b.\mathtt{isBalanced}()$ genau dann den Wert $\mathtt{true}$ hat, wenn $b$ die
      \emph{Balancierungs-Bedingung} f\"ur \emph{Heaps} erf\"ullt.  
      \hspace*{\fill} (5 Punkte)
\end{enumerate}
\vspace{0.3cm}


\noindent
\textbf{Aufgabe 8}:
Es gelte $\Sigma = \{ \mathtt{a},\,\mathtt{b},\,\mathtt{c},\,\mathtt{d},\,\mathtt{e},\,\mathtt{f} \}$.
Die H\"aufigkeit, mit der diese Buchstaben in dem zu kodierenden String $s$ auftreten, sei durch die
folgende Tabelle gegeben:

\begin{center}
\begin{tabular}[t]{|l|r|r|r|r|r|r|}
\hline
Buchstabe  & \texttt{a} & \texttt{b} & \texttt{c} & \texttt{d} & \texttt{e} & \texttt{f} \\
\hline
H\"aufigkeit &          8 &          9 &         10 &         11 &         12 &         13 \\
\hline
\end{tabular}
\end{center}
\begin{enumerate}
\item Berechnen sie einen optimalen Kodierungs-Baum f\"ur die angegebenen H\"aufigkeiten.
\item Geben die Kodierung der einzelnen Buchstaben an, die sich aus diesem Baum ergibt.
\end{enumerate}


\end{document}

%%% Local Variables: 
%%% mode: latex
%%% TeX-master: t
%%% End: 
